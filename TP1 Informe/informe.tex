\documentclass[a4paper,12pt]{article}
% es-tabla hace que se escriba "Tabla" en vez de "Cuadro"
\usepackage[spanish, es-tabla]{babel}
\usepackage[utf8]{inputenc}
\usepackage{float}
\usepackage{blindtext}

\usepackage{enumerate}
\usepackage{chngcntr}
\counterwithin*{section}{part}

% Espaciado tiene que ser 1,5
%\renewcommand{\baselinestretch}{1.5}

% Tiene que ser justificado, paquete para alinear es:
%\usepackage{ragged2e}

% Márgenes de 2 cm a ambos lados
%\usepackage[left=2cm, right=2cm]{geometry}

% Notación científica
%\providecommand{\e}[1]{\ensuremath{\times 10^{#1}}}

% Para las imagenes
\usepackage{graphicx}

% Para los apendices
\usepackage[toc,page]{appendix}

% Para que la bilbiografía sea una sección
%\usepackage{etoolbox}
%\patchcmd{\thebibliography}{\section*}{\section}{}{}

% Para que los apéndices estén en español
%\addto\captionsspanish{%
%  \renewcommand\appendixname{Anexo}
%  \renewcommand\appendixpagename{Apéndices}
%}
%\addto\captionsspanish{%
%   \renewcommand{\appendixtocname}{Apéndices}
%   \renewcommand{\appendixpagename}{Apéndices}
%}
%Fancy logo atrás de la carátula
\usepackage[pages=some,scale=2.5,angle=0,opacity=0.1]{background}
\newcommand\BackImage[2][scale=1]{%
\BgThispage
\backgroundsetup{
  contents={\includegraphics[#1]{#2}}
  }
}

\begin{document}

\pagenumbering{gobble}

\BackImage[width=.5\textwidth]{imagenes/logofiuba.png}% image on page 1
\begin{titlepage}
	\centering	
	\vspace{1cm}
	{\scshape\LARGE Facultad de ingeniería de la UBA\par}
	\vspace{1cm}
	{\scshape\Large 75.06/95.58 Organización de Datos\par}
	\vspace{1.5cm}
	{\huge\bfseries Trabajo práctico N$^\circ$1 \par Análisis de datos\par }
	\vspace{1cm}
	{Primer cuatrimestre de 2021\par}
	\vspace{2cm}
	{\Large\itshape Análisis exploratorio sobre los datos de \textit{Richter's Predictor}\par}
	\vfill

\begin{table}[H]
\label{tabla:integrantes}
\begin{tabular}{|c|c|c|}
\hline
Integrante & Padrón & Correo electrónico\\
\hline
Gutiérrez, Matías & 92172 & \texttt{matiasgutierrez@outlook.com}\\

\hline
\end{tabular}
\label{table:mediciones}
\end{table}	
	
	
	\vfill

% Bottom of the page
	{\large Nombre del grupo: \textit{DataStalkers}\\
	Link al repositorio: \texttt{git@github.com:gutierrezmatias/Organizacion-de-Datos.git}\\
	
	}

\end{titlepage}

\pagenumbering{roman}
\tableofcontents

\newpage

\pagenumbering{arabic}

\part{Introducción}
\par
En el presente informe se presentan los resultados obtenidos al realizar análisis exploratorio sobre los datos de  \textit{Richter's Predictor}. 
\par
Los datos brindados consistieron en dos datasets:
\begin{itemize}
    \item train values: Con datos relacionados a .
    \item train labels: Con datos relacionados a.
\end{itemize}

\par
Las siguientes secciones resumen los resultados obtenidos luego de realizar el análisis, junto a las conclusiones obtenidas.

\newpage

\part{Análisis previo y limpieza de datos}

\section{Introducción}

\par
...

\par ...

\paragraph{Para la proporción de valores nulos:} ....

\paragraph{Para la cantidad de valores distintos:}....

\section{Reporte detallado de la primera inspección de los datos}
....

\subsection{Dataset \textit{train values}}
\begin{itemize}
    \item ...
\end{itemize}

\subsection{Dataset \textit{train labels}}
\begin{itemize}
    \item ...
\end{itemize}

\newpage

\part{Análisis de datos}

\section{train values}
\par
...


\subsection{...}
...


\subsection{....}
\par 
..

\subsubsection*{.....}
\par
...
\begin{figure}[H]
 \centering
 \includegraphics[width=135mm, height=100mm, keepaspectratio]{graficos/...}
 \label{....}
 \caption{...}
\end{figure}

\par
....
\par 
...

\subsubsection{.....}
\par
.....
\par
.....

\begin{figure}[H]
 \centering
 \includegraphics[width=\textwidth, height=\textheight, keepaspectratio]{graficos/...}
 \caption{....}
 \label{...}
\end{figure}

...
\par
...

\newpage

\part{Conclusiones}

\section{....}
...

\section{...}
...

\section{...}
...

\section{...}
...
\end{document}